\documentclass[11pt]{article}
\usepackage{listings}
\usepackage{graphicx}
%Gummi|065|=)
\title{\textbf{Project Proposal}}
\author{Luka Filipovic\\
		20494344\\}

\begin{document}

\maketitle

My project will attempt to make an Artificially intelligent computer player for a Pokemon Trading Card Game. It will be given a specific deck, called "Night March", and will try to learn how to play it. 
$ \\ $

In doing so, Monte Carlo Methods will be used. I'm planning to use at least 2 different approaches. The first approach will be a very simple one, where some moves will be deterministic, but some moves will be stochastic, and determined by Monte Carlo. Monte Carlo will simply simulate a random game from that point for the next 3 turns or so, and then measure which move (or a set of moves) did the best, and do that move. 
$ \\ $

The second approach will be more involved. I will construct a network, where states will corrrespond to the current situation in the game (depending on cards in the hand, on the field, and currently left in the deck), and then edges will be transition functions between states, corresponding to actions taken. Due to a huge potential number of states (trillions), some assumptions and generalizations will have to be made, so that the number of states becomes a couple of million or so. The network will be stored (MongoDB might be used, for example), and it will be trained for some time, by doing random simulations of games (Monte Carlo method), and depending on the outcome giving appropriate weights to the edges, so that good actions get higher priority. $ \\ $

I will use the constructed decision network to play a game on Pokemon Trading Card Game Online system. I will play practice mode (hopefully on expert difficlty), and try to beat the computer with my own alogrithm which uses the constructed decision network. Hopefully, I will be able to win when doing a presentation. :)

\end{document}

